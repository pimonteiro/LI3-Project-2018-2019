
%%%%%%%%%%%%%%%%%%%%%%%%%%%%%%%%%%%%%%%%%%%%%%%%%%%%%%%%%%%%%%%%%%%%%%%%%%%%%%%%
%2345678901234567890123456789012345678901234567890123456789012345678901234567890
%        1         2         3         4         5         6         7         8

\documentclass[letterpaper, 10 pt, conference]{IEEEtran}  % Comment this line out
\usepackage[utf8]{inputenc}
\usepackage[T1]{fontenc}                                                         % if you need a4paper
\usepackage[portuges]{babel}
%\documentclass[a4paper, 10pt, conference]{ieeeconf}      % Use this line for a4
 \usepackage{listings}
 \lstset { %
    language=C,
}                                                         % paper
\usepackage{tikz}
\usepackage{xstring}
\usetikzlibrary{calc, arrows}
\usetikzlibrary{matrix,positioning,arrows.meta,arrows}

\tikzset{
mymat/.style={
  matrix of math nodes,
  text height=2.5ex,
  text depth=0.75ex,
  text width=3.25ex,
  align=center,
  column sep=-\pgflinewidth
  },
mymats/.style={
  mymat,
  nodes={draw,fill=#1}
  }
}
\IEEEoverridecommandlockouts                              % This command is only
                                                          % needed if you want to
                                                          % use the \thanks command
\overrideIEEEmargins
% See the \addtolength command later in the file to balance the column lengths
% on the last page of the document



% The following packages can be found on http:\\www.ctan.org
%\usepackage{graphics} % for pdf, bitmapped graphics files
%\usepackage{epsfig} % for postscript graphics files
%\usepackage{mathptmx} % assumes new font selection scheme installed
%\usepackage{times} % assumes new font selection scheme installed
%\usepackage{amsmath} % assumes amsmath package installed
%\usepackage{amssymb}  % assumes amsmath package installed

\title{\LARGE \bf
Laboratórios de Informática III - Projecto em C
}

%\author{ \parbox{3 in}{\centering Huibert Kwakernaak*
%         \thanks{*Use the $\backslash$thanks command to put information here}\\
%         Faculty of Electrical Engineering, Mathematics and Computer Science\\
%         University of Twente\\
%         7500 AE Enschede, The Netherlands\\
%         {\tt\small h.kwakernaak@autsubmit.com}}
%         \hspace*{ 0.5 in}
%         \parbox{3 in}{ \centering Pradeep Misra**
%         \thanks{**The footnote marks may be inserted manually}\\
%        Department of Electrical Engineering \\
%         Wright State University\\
%         Dayton, OH 45435, USA\\
%         {\tt\small pmisra@cs.wright.edu}}
%}

\author{Adriana Meireles (a82582)$^{1}$, Eduardo Jorge Barbosa (a83344)$^{2}$, Filipe Monteiro (puta)% <-this % stops a space
\thanks{*This work was not supported by any organization}% <-this % stops a space
\thanks{$^{1}$H. Kwakernaak is with Faculty of Electrical Engineering, Mathematics and Computer Science,
        University of Twente, 7500 AE Enschede, The Netherlands
        {\tt\small h.kwakernaak at papercept.net}}%
\thanks{$^{2}$P. Misra is with the Department of Electrical Engineering, Wright State University,
        Dayton, OH 45435, USA
        {\tt\small p.misra at ieee.org}}%
}


\begin{document}



\maketitle
\thispagestyle{empty}
\pagestyle{empty}


%%%%%%%%%%%%%%%%%%%%%%%%%%%%%%%%%%%%%%%%%%%%%%%%%%%%%%%%%%%%%%%%%%%%%%%%%%%%%%%%
\begin{abstract}

Este trabalho prático teve como objectivo a organização de grandes volumes de dados de forma a responder a questões, pré definidas, em tempo útil. Os objectos de estudo foram vários dumps do \textit{Stack Exchange}.

\end{abstract}


%%%%%%%%%%%%%%%%%%%%%%%%%%%%%%%%%%%%%%%%%%%%%%%%%%%%%%%%%%%%%%%%%%%%%%%%%%%%%%%%
\section{INTRODUCTION}

O cerne deste trabalho prático consistiu em estruturar a informação relativa a vários posts do \textit{Stack Exchange} de forma a responder em tempo útil às queries fornecidas pelos professores. Para tal, o conhecimento adquirido noutras unidades curriculares, como por exemplo Algoritmos, Programação Imperativa e Calculo de Programas, foi de extrema importância, bem como, um estudo sobre optimização de código C (linguagem exigida por esta unidade curricular). A realização deste projecto foi repartida em várias fases: análise dos dados, escolha das estruturas de dados para os representar, como fazer parse dos ficheiros, resposta as queries e por fim, optimização e limpeza do código. De forma a promover boas práticas de programação e desenvolver um workflow eficiente, todo o código desenvolvido era submetido para um repositório \text{GIT}.

\section{CONCEÇÃO DO PROBLEMA}

\subsection{ANÁLISE}

Analisando as queries fornecidas pelos professores, restringiu-se o número de ficheiros a serem lidos a 3:

\begin{itemize}
\item Users;
\item Post;
\item Tags.
\end{itemize}

Tornou-se também evidente que eram exigidos vários tipos de ordenação: por datas, por score, por número de posts, etc.  No entanto, uma grande maioria das queries tinham sempre como base um filtro por datas.

Em relação aos utilizadores tornou-se notório que bastava relacionar o seu id com resto da informação pertinente.

Finalmente,dado que as tags apareciam em várias queries foi bastante discutido como as guardar.

\section{CONCEÇÃO DA SOLUÇÃO}

Decidiu-se usar para a estrutura principal 3 \textit{Hash Tables} da \textit{GLIB} e uma estrutura com dupla ordenação \textit{ad hoc}.

As \textit{tags} ficaram agrupadas numa \textit{Hash Table} onde cada chave é a \textit{tag}, em formato \textit{String}, e o valor,  o \textit{id} correspondente.
Contemplou-se imlementar uma \textit{trie} ou uma \textit{inverted index} de forma a ser mais eficiente procurar as tags, mas dado o aumento do tempo de load, não o fizemos.

Por sua vez, os \textit{Posts} ficaram tanto numa \textit{Hash Table} como numa estrutura por nós criada, separados por \textit{Questions} e \textit{Answers}.

Os \textit{Users} ficaram numa \textit{Hash Table}.

Tanto a procura como inserção numa \textit{Hash Table} é $\mathcal{O}(1)$, e estas são as operações mais utilizadas neste projeto.
Para representar os \textit{Posts} por datas foi pensado a utilização de àrvores balanceadas. Consideramos implementar uma \textit{Red and Black Tree} dado a inserção ser menos pesada que numa \textit{AVL} mas não resolvia o problema de várias ordenações. Surgiu daí a nossa estrutura, um array de arrays com \textit{GSequences} o que permite uma ordenação por datas e uma outra ordenação à nossa escolha.

Finalmente, para realizar o parse utilizamos o parser \textit{SAX} da biblioteca \textit{libxml} dado o tamanho dos ficheiros e as operações.

\section{Parser}

Escolheu-se utilizar o parser do tipo \textit{SAX}. Este parser é orientado a eventos e recomendado pela documentação do \textit{libxml} para dar parse a grandes ficheiros. Ao contrário das duas outras alternativas este parser não constrói a árvore do xml em memória sendo muito mais rápido (cerca de 1.5x mais rápido). Um lado negativo deste parser é não ser possível editar os ficheiros xml nem percorrer várias vezes o ficheiro mas estes factores não têm qualquer relevância para este trabalho.

As outras duas alternativas eram \textit{DOM based} e \textit{DOM based} com \textit{SAX like API}.
A primeira cria a árvore do xml e carrega toda para a memória, sendo extremamente lenta além de ocupar muita memória (5x mais o tamanho do ficheiro). A sua travessia é bi orientada e não é orientada a eventos.
A última alternativa é uma versão mais evoluída do \textit{pure DOM}, sendo que não carrega a árvore toda para a memória mas sim à medida que se faz a travessia, a API também é orientada a eventos. No entanto por ir construindo a árvore é mais lenta que a \textit{SAX}.

\section{Estruturas de dados}

Utilizou-se a biblioteca \textit{GLib} para a implementação de maior parte das estruturas em cima mencionadas.
\begin{lstlisting}
struct TCD_community{
    GHashTable* profiles;
    GHashTable* posts;
    GHashTable* tags;
    TARDIS type40;
    long n_tags;
};
\end{lstlisting}

\subsection{POSTS, QUESTIONS, ANSWERS}

A representação de um \textit{Post} foi um ponto interessante deste trabalho, visto dividirem-se em \textit{Questions} e \textit{Answers}. Dado partilharem atributos, decidiu-se implementar uma \textit{union}, de forma a conseguirmos juntar os \textit{Posts} ou separá-los como nos fosse mais conveniente.
\begin{lstlisting}
struct post{
    long type; // 1 Question 2 Answer
    union content{
        QUESTION q;
        ANSWER a;
    }content;
};
\end{lstlisting}

Na representação de uma \textit{Question} além da informação básica guardamos os \textit{Ids} das respostas correspondentes. Os \textit{Ids} são inseridos via \textit{insertion sort} comparando a data de criação de cada resposta. Desta forma é possível obter todas as respostas de uma dada pergunta.
Escolheu-se utilizar um \textit{GArray} visto que o número de inserções é pequeno e desconhecido, à partida.

\begin{lstlisting}
struct question{
    MyDate creation_date;
    GArray* id_answers;
    char* title_question;
    char* tags;
    long id_question;
    long owner_id_question;
    long n_answers;
    long comments;
    int score;
};
\end{lstlisting}

\subsection{USERS}

Escolheu-se utilizar uma \textit{Hash Table} para agrupar os \textit{Users}. Cada chave é o \textit{Id} do \textit{User} e o valor é a instância que representa aquele \textit{User}.
Juntamente com os atributos que representam um utilizador guardamos numa \textit{GSequence} as questões e respostas criadas pelo dito utilizador. Escolhemos guardar desta forma em vez de um array fazendo \textit{insertion sort} pois desconhecemos o número de inserções. Além disso, é possível ordernar por vários critérios (um de cada vez).

\begin{lstlisting}
struct profile{
    GSequence* posts;
    char* about_me;
    char* name;
    long n_posts;
    long id;
    int reputation;
};
\end{lstlisting}


\subsection{TARDIS}

Com vista a resolver o problema de vários tipo de ordenação, maioritariamente em relação aos \textit{posts}, decidiu-se criar uma estrutura que consiste num array de arrays com \textit{GSequences}.
\begin{lstlisting}
struct tardis {
    GSequence*** year_questions;
    GSequence*** year_answers;
    int years;
};

\end{lstlisting}
O primeiro array corresponde aos anos, o segundo array corresponde aos meses e aos dias desse ano. O tamanho de cada array de meses e dias é sempre 31*12. Desta forma torna-se mais fácil iterar sobre a estrutura. Apesar desta forma existirem \textit{"buracos"} não são um número que justifique abandonar esta estratégia.
Cada array de meses e dias só é criado quando o primeiro \textit{Post} que uma data nessa alcance é processado, desta maneira foi possível optimizar a inserção nesta estrutura e gastar menos memória.
Com este array de arrays, temos uma granularidade de dias, meses e anos que é o pedido pelas \textit{queries} e dado os \textit{Posts} estarem numa \textit{GSequence} é possível ordenar pelo critério que nós quisermos.
Num caso extremo em que temos que ordenar um número enorme de \textit{GSequences} demora $\mathcal{O}(\log{}N)$ por \textit{GSequence}, onde o N é o número de \textit{Questions} ou \textit{Answers} (pequeno).De facto, é possível obter uma complexidade amortizada de $\mathcal{O}(1)$ nas queries que peçam Top N entre data X e data Y.
Para iterar pela TARDIS, o índice do array de anos é calculado subtraindo 2008 ao ano passado, dado que segundo a documentação do \textit{Stack Exchange} o ano mais antigo é esse. Por sua vez,os meses e dias seguem a seguinte fórmula:

Segue-se uma representação da TARDIS.
\begin{tikzpicture}[>=latex]
\matrix[mymat,anchor=west,row 2/.style={nodes=draw}]
at (0,0)
(mat1)
{
  0       &     1       &     (...)       &     7 \\
** & ** & ** & ** \\
};
\matrix[mymat,anchor=west, row 2/.style={nodes=draw}]
at (0,-3)
(mat3)
{
  0 & 1 & 2 & 3 & 4 & 5 & (..) & 371 \\
  * & * & * & * & * & * & * & * \\
};

  \node[above=0pt of mat1]
  (cella) {Array de anos};

  \node[above=-0pt of mat3]
  (cella) {Array de meses e dias};

  \node [matrix,draw] at (0,-5) (mat4){Post \\}
child {node[matrix,draw] {Post \\}}
child {node[matrix,draw] {Post \\}};

  \node[above=0pt of mat4]
  (cella) {GSequence de Posts};
\begin{scope}[shorten <= -2pt]
\draw[*->]
  (mat1-2-1.south) -- (mat3-1-1.north);
\draw[*->]
  (mat3-2-1.south) -- (mat4.north);
\end{scope}
\end{tikzpicture}



\section{CONCLUSIONS}

A conclusion section is not required. Although a conclusion may review the main points of the paper, do not replicate the abstract as the conclusion. A conclusion might elaborate on the importance of the work or suggest applications and extensions.

\addtolength{\textheight}{-12cm}   % This command serves to balance the column lengths
                                  % on the last page of the document manually. It shortens
                                  % the textheight of the last page by a suitable amount.
                                  % This command does not take effect until the next page
                                  % so it should come on the page before the last. Make
                                  % sure that you do not shorten the textheight too much.

%%%%%%%%%%%%%%%%%%%%%%%%%%%%%%%%%%%%%%%%%%%%%%%%%%%%%%%%%%%%%%%%%%%%%%%%%%%%%%%%



%%%%%%%%%%%%%%%%%%%%%%%%%%%%%%%%%%%%%%%%%%%%%%%%%%%%%%%%%%%%%%%%%%%%%%%%%%%%%%%%



%%%%%%%%%%%%%%%%%%%%%%%%%%%%%%%%%%%%%%%%%%%%%%%%%%%%%%%%%%%%%%%%%%%%%%%%%%%%%%%%
\section*{APPENDIX}

Appendixes should appear before the acknowledgment.

\section*{ACKNOWLEDGMENT}

The preferred spelling of the word ÒacknowledgmentÓ in America is without an ÒeÓ after the ÒgÓ. Avoid the stilted expression, ÒOne of us (R. B. G.) thanks . . .Ó  Instead, try ÒR. B. G. thanksÓ. Put sponsor acknowledgments in the unnumbered footnote on the first page.



%%%%%%%%%%%%%%%%%%%%%%%%%%%%%%%%%%%%%%%%%%%%%%%%%%%%%%%%%%%%%%%%%%%%%%%%%%%%%%%%

References are important to the reader; therefore, each citation must be complete and correct. If at all possible, references should be commonly available publications.



\begin{thebibliography}{99}

\bibitem{c1} G. O. Young, ÒSynthetic structure of industrial plastics (Book style with paper title and editor),Ó    in Plastics, 2nd ed. vol. 3, J. Peters, Ed.  New York: McGraw-Hill, 1964, pp. 15Ð64.
\bibitem{c2} W.-K. Chen, Linear Networks and Systems (Book style).  Belmont, CA: Wadsworth, 1993, pp. 123Ð135.
\bibitem{c3} H. Poor, An Introduction to Signal Detection and Estimation.   New York: Springer-Verlag, 1985, ch. 4.
\bibitem{c4} B. Smith, ÒAn approach to graphs of linear forms (Unpublished work style),Ó unpublished.
\bibitem{c5} E. H. Miller, ÒA note on reflector arrays (Periodical styleÑAccepted for publication),Ó IEEE Trans. Antennas Propagat., to be publised.
\bibitem{c6} J. Wang, ÒFundamentals of erbium-doped fiber amplifiers arrays (Periodical styleÑSubmitted for publication),Ó IEEE J. Quantum Electron., submitted for publication.
\bibitem{c7} C. J. Kaufman, Rocky Mountain Research Lab., Boulder, CO, private communication, May 1995.
\bibitem{c8} Y. Yorozu, M. Hirano, K. Oka, and Y. Tagawa, ÒElectron spectroscopy studies on magneto-optical media and plastic substrate interfaces(Translation Journals style),Ó IEEE Transl. J. Magn.Jpn., vol. 2, Aug. 1987, pp. 740Ð741 [Dig. 9th Annu. Conf. Magnetics Japan, 1982, p. 301].
\bibitem{c9} M. Young, The Techincal Writers Handbook.  Mill Valley, CA: University Science, 1989.
\bibitem{c10} J. U. Duncombe, ÒInfrared navigationÑPart I: An assessment of feasibility (Periodical style),Ó IEEE Trans. Electron Devices, vol. ED-11, pp. 34Ð39, Jan. 1959.
\bibitem{c11} S. Chen, B. Mulgrew, and P. M. Grant, ÒA clustering technique for digital communications channel equalization using radial basis function networks,Ó IEEE Trans. Neural Networks, vol. 4, pp. 570Ð578, July 1993.
\bibitem{c12} R. W. Lucky, ÒAutomatic equalization for digital communication,Ó Bell Syst. Tech. J., vol. 44, no. 4, pp. 547Ð588, Apr. 1965.
\bibitem{c13} S. P. Bingulac, ÒOn the compatibility of adaptive controllers (Published Conference Proceedings style),Ó in Proc. 4th Annu. Allerton Conf. Circuits and Systems Theory, New York, 1994, pp. 8Ð16.
\bibitem{c14} G. R. Faulhaber, ÒDesign of service systems with priority reservation,Ó in Conf. Rec. 1995 IEEE Int. Conf. Communications, pp. 3Ð8.
\bibitem{c15} W. D. Doyle, ÒMagnetization reversal in films with biaxial anisotropy,Ó in 1987 Proc. INTERMAG Conf., pp. 2.2-1Ð2.2-6.
\bibitem{c16} G. W. Juette and L. E. Zeffanella, ÒRadio noise currents n short sections on bundle conductors (Presented Conference Paper style),Ó presented at the IEEE Summer power Meeting, Dallas, TX, June 22Ð27, 1990, Paper 90 SM 690-0 PWRS.
\bibitem{c17} J. G. Kreifeldt, ÒAn analysis of surface-detected EMG as an amplitude-modulated noise,Ó presented at the 1989 Int. Conf. Medicine and Biological Engineering, Chicago, IL.
\bibitem{c18} J. Williams, ÒNarrow-band analyzer (Thesis or Dissertation style),Ó Ph.D. dissertation, Dept. Elect. Eng., Harvard Univ., Cambridge, MA, 1993.
\bibitem{c19} N. Kawasaki, ÒParametric study of thermal and chemical nonequilibrium nozzle flow,Ó M.S. thesis, Dept. Electron. Eng., Osaka Univ., Osaka, Japan, 1993.
\bibitem{c20} J. P. Wilkinson, ÒNonlinear resonant circuit devices (Patent style),Ó U.S. Patent 3 624 12, July 16, 1990.






\end{thebibliography}




\end{document}



