\documentclass[letterpaper, 10 pt, conference]{IEEEtran} % Comment this line out
\usepackage[utf8]{inputenc}
\usepackage[T1]{fontenc}                                                         % if you need a4paper
\usepackage[portuges]{babel}
%\documentclass[a4paper, 10pt, conference]{ieeeconf}      % Use this line for a4
\usepackage{listings}
\lstset { %
language=Java,
}                                                         % paper
\usepackage{tikz}
\usepackage{xstring}
\usetikzlibrary{calc, arrows}
\usetikzlibrary{matrix,positioning,arrows.meta,arrows}

\tikzset{
mymat/.style={
matrix of math nodes,
text height=2.5ex,
text depth=0.75ex,
text width=3.25ex,
align=center,
column sep=-\pgflinewidth
},
mymats/.style={
mymat,
nodes={draw,fill=#1}
}
}
\IEEEoverridecommandlockouts                              % This command is only
                                                      % needed if you want to
                                                      % use the \thanks command
\overrideIEEEmargins
% See the \addtolength command later in the file to balance the column lengths
% on the last page of the document



% The following packages can be found on http:\\www.ctan.org
%\usepackage{graphics} % for pdf, bitmapped graphics files
%\usepackage{epsfig} % for postscript graphics files
%\usepackage{mathptmx} % assumes new font selection scheme installed
%\usepackage{times} % assumes new font selection scheme installed
%\usepackage{amsmath} % assumes amsmath package installed
%\usepackage{amssymb}  % assumes amsmath package installed

\title{\LARGE \bf
Laboratórios de Informática III - Projecto em Java
}

%\author{ \parbox{3 in}{\centering Huibert Kwakernaak*
%         \thanks{*Use the $\backslash$thanks command to put information here}\\
%         Faculty of Electrical Engineering, Mathematics and Computer Science\\
%         University of Twente\\
%         7500 AE Enschede, The Netherlands\\
%         {\tt\small h.kwakernaak@autsubmit.com}}
%         \hspace*{ 0.5 in}
%         \parbox{3 in}{ \centering Pradeep Misra**
%         \thanks{**The footnote marks may be inserted manually}\\
%        Department of Electrical Engineering \\
%         Wright State University\\
%         Dayton, OH 45435, USA\\
%         {\tt\small pmisra@cs.wright.edu}}
%}

\author{Adriana Meireles (a82582), Eduardo Jorge Barbosa (a83344), Filipe Monteiro (a80229)% <-this % stops a space
}

%%%%%%%%%%%%%%%%%%%%%%%%%%%%%%%%%%%%%%%%%%%%%%%%%%%%%%%%%%%%%%%%%%%%%%%%%%%%%%%%%%%%%%%%%%%%%%%%%%%%%%%%%%%%%%%%%%%%%%%%%%%%%%%%%%%%%%%%%%%%%%%%%%%%%%%%%%%%%%%%%%%%%%%%%%%%%%%%%%%%%%
\begin{document}



\maketitle
\thispagestyle{empty}
\pagestyle{empty}


%%%%%%%%%%%%%%%%%%%%%%%%%%%%%%%%%%%%%%%%%%%%%%%%%%%%%%%%%%%%%%%%%%%%%%%%%%%%%%%%%%%%%%%%%%%%%%%%%%%%%%%%%%%%%%%%%%%%%%%%%%%%%%%%%%%%%%%%%%%%%%%%%%%%%%%%%%%%%%%%%%%%%%%%%%%%%%%%%%%%%%
\begin{abstract}

O objectivo desta parte do trabalho foi re-implementar a solução desenvolvida em \textit{C} num paradigma totalmente orientado a objectos em \textit{Java}. Apesar de seguir a arquitectura anterior não é um factor que limite esta fase, sendo que \textit{Java} proporciona um número enorme de estruturas já implementadas através das \textit{Java Collections}.

\end{abstract}


%%%%%%%%%%%%%%%%%%%%%%%%%%%%%%%%%%%%%%%%%%%%%%%%%%%%%%%%%%%%%%%%%%%%%%%%%%%%%%%%%%%%%%%%%%%%%%%%%%%%%%%%%%%%%%%%%%%%%%%%%%%%%%%%%%%%%%%%%%%%%%%%%%%%%%%%%%%%%%%%%%%%%%%%%%%%%%%%%%%%%%
\section{INTRODUÇÃO}

Irá ser abordado neste relatório a arquitectura da nossa solução nomeadamente estruturas usadas, estratégias utilizadas nas \textit{queries}, encapsulamento e o modelo \textit{MVC}.

%%%%%%%%%%%%%%%%%%%%%%%%%%%%%%%%%%%%%%%%%%%%%%%%%%%%%%%%%%%%%%%%%%%%%%%%%%%%%%%%%%%%%%%%%%%%%%%%%%%%%%%%%%%%%%%%%%%%%%%%%%%%%%%%%%%%%%%%%%%%%%%%%%%%%%%%%%%%%%%%%%%%%%%%%%%%%%%%%%%%%%

\section{PARSER}

Como o \textit{parse} usado na primeira fase existe por \textit{default} em \textit{Java}, escolheu-se mais uma vez utilizar o parser do tipo \textit{SAX}.
É do notar a diferença entre a implementação em \textit{C} e em \textit{Java}. Utilizando uma implementação análoga do projecto original, o tempo de load aumentou cerca de \textbf{4 a 8 milisegundos}.

\subsection{ALTERNATIVAS}

Um \textit{parser} alternativo que surgiu foi o \textit{StAX}.
Tal como o \textit{Sax} este \textit{parser} é orientado a eventos. A diferença entre os dois encontra-se no facto do \textit{Sax} ser \textit{push based} e o \textit{StAX} ser \textit{pull based}. De facto, o \textit{StAX} pode ser visto como uma espécie de iterador que gera os eventos que o \textit{parser handler} tem que lidar, já o \textit{sax} recebe os eventos e lida com eles.
Do que foi pesquisado \textit{StAX} não oferece nenhuma vantagem sobre \textit{sax} e sendo que para a parte de \textit{C} já foi usado \textit{sax} mantivemos.

%%%%%%%%%%%%%%%%%%%%%%%%%%%%%%%%%%%%%%%%%%%%%%%%%%%%%%%%%%%%%%%%%%%%%%%%%%%%%%%%%%%%%%%%%%%%%%%%%%%%%%%%%%%%%%%%%%%%%%%%%%%%%%%%%%%%%%%%%%%%%%%%%%%%%%%%%%%%%%%%%%%%%%%%%%%%%%%%%%%%%%

\section{MODULARIDADE}

\subsection{ENGINE}
A nossa arquitetura segue este diagrama:

\begin{figure}[h!]
  \centering
  \includegraphics[width=0.5\textwidth]{engine.png}
   \caption{Classes principais}
\end{figure}

\begin{figure}[h!]
  \centering
  \includegraphics[width=0.5\textwidth]{engine2.png}
   \caption{Diagrama de dependências}
\end{figure}

Os utilizadores são guardados na classe \textit{Profile}. Questões e respostas são guardadas nas classes \textit{Question} e \textit{Answer} respectivamente que estendem a classe \textit{Post}. Entra outras estruturas os \textit{Posts} são guardados na \textit{Tardis}.
A classe \textit{MainStruct} agrega todos os dados. Por fim, a classe \textit{TCDCommunity} implementa a \textit{interface} com as \textit{queries}.
Todas as \textit{queries} foram resolvidas na sua respectiva classe sendo que cada classe apenas possui um método estático (que resolve a questão).

\subsection{LI3}

Com base numa estruturação MVC, tentamos criar um menu simples, onde se pode testar cada query independentemente. Foi baseado no pdf colocado na página da cadeira,
com modificações para este contexto, assim como acréscimo de opções.
Por isso temos as seguintes classes:
\begin{itemize}
    \item \textit{StackOverflowMVCApp}: main do programa. Aqui fazemmos a lógica de chamar as queries independentes, ou executar a main dada pelos professores que corre todas as queries de uma vez, apresntando tempos e resultados;
    \item \textit{StackOverflowView}: onde tem as views do programa (estáticas), os menus;
    \item \textit{StackOverflowController}: controlador do flow do programa. Aqui fazem-se as decisões de menus a apresentar, chamadas das queries;
    \item \textit{Opção,Menu,Menus}: opçcão, menu e conjunto de menus, respetivamente.
\end{itemize}

\section{Classes e Estrutura}

Em seguida irá ser feita uma descrição das estruturas utilizadas para efeitos de contextualização. Além do mais são \textbf{discutidas} certas decisões.

\subsection{MAIN STRUCT}
À semelhança do projecto em \textit{C}, a estrutura principal é constituída por 3 \textit{HashMaps}, uma para os perfis, outra para \textit{posts} e outra para as \textit{tags}, e pela \textit{Tardis} que permite reaver os \textit{Posts} organizados por qualquer critério.
\newline

\begin{figure}[h!]
  \centering
  \includegraphics[width=0.3\textwidth]{estruturas.png}
   \caption{Estruturas principais}
\end{figure}

\subsection{USER}
A seguinte informação é guardada para cada utilizador:
\newline

\begin{figure}[h!]
  \centering
  \includegraphics[width=0.3\textwidth]{profile.png}
   \caption{User}
\end{figure}



\subsubsection{POSTS}


Todos os \textit{posts} do utilizador são guardados num \textit{TreeSet} que segue a ordem natural dos mesmos (cronológica).
Foi debatido guardar todos os \textit{posts} ou apenas os últimos 10 e caso fosse esse o cenário se eram guardados num \textit{BoundedTreeSet} ou numa \textit{max heap}.
Um \textit{BoundedTreeSet} tem a vantagem de ser de relativamente fácil implementação mas a ordenação é mais custosa que uma \textit{max heap}.
No fim decidimos guardar todos os \textit{posts} do utilizador visto terem utilidade para outras \textit{queries} nomeadamente a \textit{11}.

\subsubsection{COMPARABLE}

Foi discutido implementar uma ordem natural para os \textit{users} mas visto que apenas é necessária em duas \textit{queries} e por critérios diferentes (reputação e número de posts), não foi feito.

\subsection{POST, QUESTION, ANSWER}

\subsubsection{POST}
Como classe \textit{mae} temos a \textit{Post} que guardas as informações comuns tanto a questões como respostas. Esta classe implementa a \textit{interface comparable} forçando a ordem natural ser cronológica (critério mais comum encontrado por nós).

\begin{figure}[h!]
  \centering
  \includegraphics[width=0.3\textwidth]{post.png}
   \caption{Post}
\end{figure}

\subsubsection{ANSWER}

As respostas apenas acrescentam um campo do tipo \textit{long} que corresponde ao \textit{id} da questão a que se refere.

\subsubsection{QUESTION}
\begin{figure}[h!]
  \centering
  \includegraphics[width=0.3\textwidth]{question.png}
   \caption{Question}
\end{figure}

Com o objectivo de inter-ligar as respostas com as questões decidiu-se guardar num \textit{Map} todas as respostas relativas a uma dada questão onde a chave é o \textit{id} respectivo.

\subsection{TARDIS}

A \textit{TARDIS} foi a estrutura que mais alterações sofreu.
Enquanto que na implementação em \textit{C} consistia em duas "variaveis de classe", decidiu-se implementar apenas uma do tipo genérico \textit{Post}. Esta mudança deu-se ao facto de não se ter ganho quase nada com a separação de perguntas e respostas.
Outra mudança significativa foram as estruturas usadas para criar a \textit{TARDIS} em si Enquanto que no projecto em \textit{C} foram usados \textit{Arrays} de \textit{Arrays} para \textit{GSequences} nesta implementação foram usados \textit{HashMap} para \textit{HashMap} para \textit{List<Post>}. Cada chave corresponde a um indice (ou de anos ou de meses/dias).

\begin{figure}[h!]
  \centering
  \includegraphics[width=0.5\textwidth]{tardisVar.png}
   \caption{Tardis}
\end{figure}

\subsubsection{ANAMORFISMO}

De forma a obter \textit{posts} dentro de um intrevalo de tempo e ordenados por um critério \textit{ad hoc}  foi implementado o metodo \textit{getBetweenBy}:

\begin{figure}[h!]
  \centering
  \includegraphics[width=0.5\textwidth]{tardisMetodos.png}
   \caption{Metodos da Tardis}
\end{figure}


Em primeiro lugar é interessante notar a assinatura do metódo \textit{TreeSet<? extends Post>}. Restringimos o tipo de retorno com uma \textit{bounded wildcard} sendo assim possível retornar tanto um \textit{TreeSet} de \textit{Post}, \textit{Answer} ou \texit{Question}.
Os seus parâmetros também são bastante interessantes. De forma a saber que tipo procurar (ou tipos), passamos a Class de procura como um genérico \texit{Class<E> type} onde o \textit{genérico E} é \textit{restringido} na assinatura do método \textit{<E extends Post>}.
Dado a falta de \textit{function pointers} (tanto quanto sabemos) passamos um \texit{Comparator} para ordenar o \texit{TreeSet}.

\begin{figure}[h!]
  \centering
  \includegraphics[width=0.5\textwidth]{ana.png}
   \caption{Stream da vida}
\end{figure}


Este método tem como gene uma stream. Isto torna o raciocinio muito mais simples do que com \textit{forEach} ou iteradores externos.
De facto, este método é um anamorfismo de \textit{TreeSets}.

12


\subsection{COMPARATORS --- EXCEPÇÕES}

Para comparadores, temos os seguintes:
\begin{itemize}
    \item PostCreationDateComparator --> Compara dois Posts em relação á sua data de criação, de forma \textbf{descendente};
    \item QuestionCreationDateComparator --> Compara duas Question em relação á sua data de criação, de forma \textbf{descendente}\footnote{Apesar de já existir um comparador igual para Posts, este foi necessário para o uso da Tardis, em alguns métodos.};
    \item ProfileNPostsComparator --> Compara dois Utilizadores em relacão ao número de posts, de forma \textbf{descendente};
    \item AnswserScoreComparator --> Compara duas Answer em relação ao seu score, de forma \textbf{descendente};
    \item QuestionAnswerCountComparator --> Compara duas Question em relação ao número de respostas, de forma \textbf{descendente}.
\end{itemize}



Quanto a excepções, criamos as seguintes:
\begin{itemize}
     \item NoPostFoundException --> Caso um ID nao exista entre os Posts;
     \item NoProfileFoundException --> Caso um ID nao exista entre os Perfis;
     \item NoTagFoundException --> Caso um ID nao exista entre as Tags;
     \item PostIsNotOfRightTypeException --> Caso uma query exiga algo sobre um tipo de Post, mas recebe outro.
\end{itemize}

%%%%%%%%%%%%%%%%%%%%%%%%%%%%%%%%%%%%%%%%%%%%%%%%%%%%%%%%%%%%%%%%%%%%%%%%%%%%%%%%%%%%%%%%%%%%%%%%%%%%%%%%%%%%%%%%%%%%%%%%%%%%%%%%%%%%%%%%%%%%%%%%%%%%%%%%%%%%%%%%%%%%%%%%%%%%%%%%%%%%%%

\section{ESTRUTURAÇÃO DE QUERYS}

\subsection{QUERY1}

Nesta query começamos por verificar o tipo do \textit{Post} pedido. Caso seja uma \textit{Answer} vai buscar a informação
à \textit{Question} a que pertence, caso contrário vai diretamente a esta.
É importante referir que alteramos a interface \negrito{TADCommunity}, colocando na assinatura desta query a
possibilidade de produzir \negrito{excepções} caso: não haja o Post ou o utilizador a que pertence, a Question não existe.


\subsection{QUERY2}

Aqui inserimos todos os perfis existentes na estrutura principal num TreeSet, usando o comparador \textit{ProfileNPostsComparator}
que ordena perfis consoante o nº de posts deste(descendente). Depois retorna apenas os primeiros N
perfis, sob a forma de uma lista de longs (usando streams, maps e limites).

\subsection{QUERY3}

Como a nossa estrutura Tardis possui um método que retorna todos os Posts existentes no programa, apenas chamamos esse método,
retornando um par apenas com o tamanho de cada tipo de Post(Question e Answer).

\subsection{QUERY4}

Nesta query começamos por recolher todas as Question entre as datas pedidas, através da Tardis(ordenadas por data).
Depois retornamos uma lista com o ID de todas as Question da Tardis.
\newline
É importante referir que experimentamos a implementação com iteradores externos e internos, estando os resultados mais à frente
no relatório.

\subsection{QUERY5}

Aqui acedemos ao HashMap da estrutura principal para receber o Profile com o ID passado. Depois retornamos um par com a bio deste Profile e os
10 posts mais recentes(através do TreeSet<Post> existente no Profile, apenas pegando os 10 primeiros).
Fizemos também a experiência de ir buscar os 10 últimos posts deste (sendo os 10 primeiros no TreeSet do Profile) com
iteradores internos e externos, mais à frente apresentando resultados.

\subsection{QUERY6}

Usando a Tardis para ter todas as Answers entre as datas pretendidas, ordenadas por um comparator \textint{AnswerScoreComparator}
que ordena por score (votos - descendente). Depois usamos um iterador externo para percorrer apenas as primeiras N Answer ordenadas.

\subsection{QUERY7}

É importante referir que tomamos a estratégia onde tanto as perguntas como as respostas têm de estar entre as datas pedidas.
Por isso, começamos por recolher todas as Question compreendidas entre as datas pedidas, onde depois, percorremos cada uma contando quantas respostas tem também entre estas datas,
adicionando a um par constituído pelo número de respostas calculado e o ID da Question. Adicionamos estes pares
a uma TreeSet que ordena os pares por ordem decrescente da sua primeira componente (número de respostas),
retornando assim a segunda componente do primeiro par do Set.

\subsection{QUERY8}

Percorrendo todas as Question (que possuem título), recorrendo à Tardis, adicionamos as primeiras N Question ,o ID onde a palavra
existe no título(usando iteradores externos).

\subsection{QUERY9}

Nesta query pegamos em todos os Post do primeiro utilizador passado e fazemos o seguinte flow:
\begin{itemize}
    \item Se o Post for uma Answer, verificamos se a Question a que pertence a resposta pertence ao segundo utiizador. Se sim,
    inserimos num TreeSet ordenado por data(PostCreationDateComparator) caso ainda não exista neste;
    \item Se for uma Question, percorremos todas as respostas, verificando se alguma pertence ao segundo utilizador. Se sim, inserimos
    no TreeSet caso ainda nao exista.
\end{itemize}
    Importante referir que: não vale a pena continuar a percorrer as respostas ou questoes quando já se encontrou algo em comum.


\subsection{QUERY10}

Nesta começamos por ir buscar o Post a que pertence o ID. Se não for uma Question atira uma excepção \textit{PostIsNotOfRightType}
, indicando que não se pode obter as respostas de uma reposta.
Depois percorremos cada resposta, calculando o score desta e atribuindo a um par constituído por este e pelo ID da resposta.
Caso algum score seja superior ao existente no par, atualiza-se este, retornando no fim a parte do par com o ID da reposta.
Para facilitar, criamos um método que calcula o score de uma Answer.

\subsection{QUERY11}

TODO

\section{Conclusão}

Esta segunda fase do projeto deu para aprender como deve ser feito um trabalho em Java, no entanto o processamento foi mais lento relativamente ao trabalho em C.Também devem ser implementadas estruturas eficientes,guardar apenas o necessário e não fazer cópias desnecessárias.
\newline
Em comparação à primeira fase, esta foi mais fácil.Não tivemos que criar tabelas de Hash ou listas ligadas,visto que o Java contém muitas estruturas pré definidas eficientes e relativamente mais fáceis de usar.Também não houve necessidade de libertar memória alocada, visto que o garbage collector faz isso por nós.
\newline
Também podemos aprender a usar streams do Java 8, apesar de serem ligeiramente mais lentas que os iteradores externos,tornam o código mais compacto,facilitando a leitura e o debug.
\newline
Tal como na primeira fase,consideramos que cumprimos os objetivos pretendidos.











%%%%%%%%%%%%%%%%%%%%%%%%%%%%%%%%%%%%%%%%%%%%%%%%%%%%%%%%%%%%%%%%%%%%%%%%%%%%%%%%%%%%%%%%%%%%%%%%%%%%%%%%%%%%%%%%%%%%%%%%%%%%%%%%%%%%%%%%%%%%%%%%%%%%%%%%%%%%%%%%%%%%%%%%%%%%%%%%%%%%%%

\addtolength{\textheight}{-12cm}   % This command serves to balance the column lengths
                              % on the last page of the document manually. It shortens
                              % the textheight of the last page by a suitable amount.
                              % This command does not take effect until the next page
                              % so it should come on the page before the last. Make
                              % sure that you do not shorten the textheight too much.

%%%%%%%%%%%%%%%%%%%%%%%%%%%%%%%%%%%%%%%%%%%%%%%%%%%%%%%%%%%%%%%%%%%%%%%%%%%%%%%%



%%%%%%%%%%%%%%%%%%%%%%%%%%%%%%%%%%%%%%%%%%%%%%%%%%%%%%%%%%%%%%%%%%%%%%%%%%%%%%%%



%%%%%%%%%%%%%%%%%%%%%%%%%%%%%%%%%%%%%%%%%%%%%%%%%%%%%%%%%%%%%%%%%%%%%%%%%%%%%%%%
%\section*{APPENDIX}

%Appendixes should appear before the acknowledgment.

%\section*{ACKNOWLEDGMENT}

%The preferred spelling of the word ÒacknowledgmentÓ in America is without an ÒeÓ after the ÒgÓ. Avoid the stilted expression, ÒOne of us (R. B. G.) thanks . . .Ó  Instead, try ÒR. B. G. thanksÓ. Put sponsor acknowledgments in the unnumbered footnote on the first page.



%%%%%%%%%%%%%%%%%%%%%%%%%%%%%%%%%%%%%%%%%%%%%%%%%%%%%%%%%%%%%%%%%%%%%%%%%%%%%%%%

%References are important to the reader; therefore, each citation must be complete and correct. If at all possible, references should be commonly available publications.



%\begin{thebibliography}{99}
%\bibitem{c1} G. O. Young, ÒSynthetic structure of industrial plastics (Book style with paper title and editor),Ó    in Plastics, 2nd ed. vol. 3, J. Peters, Ed.  New York: McGraw-Hill, 1964, pp. 15Ð64.
%\bibitem{c2} W.-K. Chen, Linear Networks and Systems (Book style).  Belmont, CA: Wadsworth, 1993, pp. 123Ð135.

%\end{thebibliography}
\end{document}
