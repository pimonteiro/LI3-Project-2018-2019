\documentclass[letterpaper, 10 pt, conference]{IEEEtran}  % Comment this line out
\usepackage[utf8]{inputenc}
\usepackage[T1]{fontenc}                                                         % if you need a4paper
\usepackage[portuges]{babel}
%\documentclass[a4paper, 10pt, conference]{ieeeconf}      % Use this line for a4
 \usepackage{listings}
 \lstset { %
    language=Java,
}                                                         % paper
\usepackage{tikz}
\usepackage{xstring}
\usetikzlibrary{calc, arrows}
\usetikzlibrary{matrix,positioning,arrows.meta,arrows}

\tikzset{
mymat/.style={
  matrix of math nodes,
  text height=2.5ex,
  text depth=0.75ex,
  text width=3.25ex,
  align=center,
  column sep=-\pgflinewidth
  },
mymats/.style={
  mymat,
  nodes={draw,fill=#1}
  }
}
\IEEEoverridecommandlockouts                              % This command is only
                                                          % needed if you want to
                                                          % use the \thanks command
\overrideIEEEmargins
% See the \addtolength command later in the file to balance the column lengths
% on the last page of the document



% The following packages can be found on http:\\www.ctan.org
%\usepackage{graphics} % for pdf, bitmapped graphics files
%\usepackage{epsfig} % for postscript graphics files
%\usepackage{mathptmx} % assumes new font selection scheme installed
%\usepackage{times} % assumes new font selection scheme installed
%\usepackage{amsmath} % assumes amsmath package installed
%\usepackage{amssymb}  % assumes amsmath package installed

\title{\LARGE \bf
Laboratórios de Informática III - Projecto em Java
}

%\author{ \parbox{3 in}{\centering Huibert Kwakernaak*
%         \thanks{*Use the $\backslash$thanks command to put information here}\\
%         Faculty of Electrical Engineering, Mathematics and Computer Science\\
%         University of Twente\\
%         7500 AE Enschede, The Netherlands\\
%         {\tt\small h.kwakernaak@autsubmit.com}}
%         \hspace*{ 0.5 in}
%         \parbox{3 in}{ \centering Pradeep Misra**
%         \thanks{**The footnote marks may be inserted manually}\\
%        Department of Electrical Engineering \\
%         Wright State University\\
%         Dayton, OH 45435, USA\\
%         {\tt\small pmisra@cs.wright.edu}}
%}

\author{Adriana Meireles (a82582), Eduardo Jorge Barbosa (a83344), Filipe Monteiro (a80229)% <-this % stops a space
}


\begin{document}

\maketitle
\thispagestyle{empty}
\pagestyle{empty}


%%%%%%%%%%%%%%%%%%%%%%%%%%%%%%%%%%%%%%%%%%%%%%%%%%%%%%%%%%%%%%%%%%%%%%%%%%%%%%%%
\begin{abstract}

Este trabalho prático teve como objectivo replicar o anterior projeto (aplicado em C) mas agora em Java - a organização
de grandes volumes de dados de forma a responder a questões, pré definidas, em tempo útil. Os objectos de estudo foram vários dumps do \textit{Stack Exchange}.

\end{abstract}

%%%%%%%%%%%%%%%%%%%%%%%%%%%%%%%%%%%%%%%%%%%%%%%%%%%%%%%%%%%%%%%%%%%%%%%%%%%%%%%%
\section{INTRODUÇÃO}

O fundamental deste trabalho prático consistiu em converter as existente estruturas e querys construidas no anterior projeto,
em estruturas mais dinâmicas, de mais flexibilidade e portabilidade, assim como melhoras de desempenho por parte de
\textit{features} existentes na linguagem de programação Java.

\section{ESTRUTURAS DE DADOS}

\subsection{MAIN STRUCT}

\begin{lstlisting}
public class Main_Struct {

private HashMap<Long,Profile> profiles;
private HashMap<Long, Post> posts;
private HashMap<String,Long> tags;
private Tardis tardis64;

}
\end{lstlisting}

Tal como no projeto em C, a estrutura principal é constituida por 3 HashMaps, uma para os perfis, outra para posts
e outra para tags, e ainda a reformulado Tardis que contem os posts todos organizados pelo critério \textit{x}.


\subsection{USER}

Na Class de um Utilizador existem, entre os dados pessoais (nome, bio, ID, etc) uma TreeSet de Posts
pertencentes ao utilizador (ordenados naturalmente do mais recente - mais antigo).
Sendo uma Class tão básica, não foi necessário acrescentar extras para possivelmente melhorar desempenhos.

\begin{lstlisting}
public class Profile{
    private TreeSet<Post> posts;
    private String about_me;
    private String name;
    private long n_posts;
    private long id;
    private int reputation;
}
\end{lstlisting}

Nesta foram implementados apenas os métodos básicos (construtores, gets, sets, clone, toString).

\subsection{POST, QUESTION, ANSWER}

A Class Post é uma Class Abstracta de onde duas classes extendem: Question e Answer.
Foi utilizada uma class Abstrata para podermos tratar destas duas subclasses como se fossem
"do mesmo tipo", facilitando assim o uso destas.

Class Post:
\begin{lstlisting}
public abstract class Post {

    private long type;
}

public class Question extends Post {

    private LocalDateTime creation_date;
    private HashMap<Long,Answer> answers;
    private String title;
    private List<String> tags; //TODO Maybe we can change with java
    private long id;
    private long owner_id;
    private long n_answers;
    private long comments;
    private long score;
}

public class Answer extends Post {

    private LocalDateTime creation_date;
    private long id;
    private long owner_id;
    private long parent_id;
    private long comments;
    private long score;

}
\end{lstlisting}
Esta possui também um método compARE FALTA VER ISTO.

\subsection{TARDIS}
//eduardo



\subsection{ESTRUTURAS AUXILIARES && EXCEPÇÕES}
Para facilitar o manuseamento de algumas querys, decidiu-se criar e/ou recriar algumas classes já implementadas
pelo Java, mas com pequenos acréscimos. Nomeadamente:
\subsubsection{BoundedTreeSet}
\begin{lstlisting}
public class BoundedTreeSet<E> extends TreeSet<E> {
    private int limit;

    @Override
    public boolean add(E e){
        super.add(e);
        if(this.size() > this.limit){
        this.remove(this.last());
        }
        return true;
    }
}
\end{lstlisting}
Esta TreeSet, funciona exatamente como uma TreeSet habitual, perfeita para inserções ordenadas, mas tem
uma informação e cuidado extra: um limite de tamanho. Sempre que estiver no tamanho máximo insere ordenadamente
e retira o último, mantendo sempre o seu tamanho controlado. Isto foi nos útil particularmente na \underline{Query 5}.


Em relação a comparadores, temos as seguintes:
\begin{itemize}
    \item PostCreationDateComparator --> Compara dois Posts em relação á sua data de criação, de forma \textbf{Descendente}.
    \item ProfileNPostsComparator --> Compara dois Utilizadores em relacão ao número de posts, de forma \textbf{Descendente}.
    \item QuestionCreationDateComparator --> Compara duas Question em relação á sua data de criação, de forma \textbf{Descendente}.
\end{itemize}


Quanto a excepções, criamos as seguintes:
\begin{itemize}
    \item NoPostFoundException --> Caso um ID nao exista entre os Posts.
    \item NoProfileFoundException --> Caso um ID nao exista entre os Perfis.
    \item NoTagFoundException --> Caso um ID nao exista entre as Tags. SUJEITO A TROCA
\end{itemize}
%%%%%%%%%%%%%%%%%%%%%%%%%%%%%%%%%%%%%%%%%%%%%%%%%%%%%%%%%%%%%%%%%%%%%%%%%%%%%%%%%%%%%%%%%%
\subsection{BENCHMARKS}

\end{document}