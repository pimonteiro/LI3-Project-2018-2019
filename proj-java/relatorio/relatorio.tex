%%%%%%%%%%%%%%%%%%%%%%%%%%%%%%%%%%%%%%%%%%%%%%%%%%%%%%%%%%%%%%%%%%%%%%%%%%%%%%%%
%2345678901234567890123456789012345678901234567890123456789012345678901234567890
%        1         2         3         4         5         6         7         8

\documentclass[letterpaper, 10 pt, conference]{IEEEtran}  % Comment this line out
\usepackage[utf8]{inputenc}
\usepackage[T1]{fontenc}                                                         % if you need a4paper
\usepackage[portuges]{babel}
%\documentclass[a4paper, 10pt, conference]{ieeeconf}      % Use this line for a4
 \usepackage{listings}
 \lstset { %
    language=Java,
}                                                         % paper
\usepackage{tikz}
\usepackage{xstring}
\usetikzlibrary{calc, arrows}
\usetikzlibrary{matrix,positioning,arrows.meta,arrows}

\tikzset{
mymat/.style={
  matrix of math nodes,
  text height=2.5ex,
  text depth=0.75ex,
  text width=3.25ex,
  align=center,
  column sep=-\pgflinewidth
  },
mymats/.style={
  mymat,
  nodes={draw,fill=#1}
  }
}
\IEEEoverridecommandlockouts                              % This command is only
                                                          % needed if you want to
                                                          % use the \thanks command
\overrideIEEEmargins
% See the \addtolength command later in the file to balance the column lengths
% on the last page of the document



% The following packages can be found on http:\\www.ctan.org
%\usepackage{graphics} % for pdf, bitmapped graphics files
%\usepackage{epsfig} % for postscript graphics files
%\usepackage{mathptmx} % assumes new font selection scheme installed
%\usepackage{times} % assumes new font selection scheme installed
%\usepackage{amsmath} % assumes amsmath package installed
%\usepackage{amssymb}  % assumes amsmath package installed

\title{\LARGE \bf
Laboratórios de Informática III - Projecto em Java
}

%\author{ \parbox{3 in}{\centering Huibert Kwakernaak*
%         \thanks{*Use the $\backslash$thanks command to put information here}\\
%         Faculty of Electrical Engineering, Mathematics and Computer Science\\
%         University of Twente\\
%         7500 AE Enschede, The Netherlands\\
%         {\tt\small h.kwakernaak@autsubmit.com}}
%         \hspace*{ 0.5 in}
%         \parbox{3 in}{ \centering Pradeep Misra**
%         \thanks{**The footnote marks may be inserted manually}\\
%        Department of Electrical Engineering \\
%         Wright State University\\
%         Dayton, OH 45435, USA\\
%         {\tt\small pmisra@cs.wright.edu}}
%}

\author{Adriana Meireles (a82582), Eduardo Jorge Barbosa (a83344), Filipe Monteiro (a80229)% <-this % stops a space
}


\begin{document}



\maketitle
\thispagestyle{empty}
\pagestyle{empty}


%%%%%%%%%%%%%%%%%%%%%%%%%%%%%%%%%%%%%%%%%%%%%%%%%%%%%%%%%%%%%%%%%%%%%%%%%%%%%%%%
\begin{abstract}

A segunda fase do trabalho prático de LI3 será realizada em Java e o cenário é exactamente o mesmo tratado no projecto em C. ou seja, serão processados exactamente os mesmos ficheiros XML e dadas respostas exactamente às mesmas interrogações.

\end{abstract}


%%%%%%%%%%%%%%%%%%%%%%%%%%%%%%%%%%%%%%%%%%%%%%%%%%%%%%%%%%%%%%%%%%%%%%%%%%%%%%%%
\section{INTRODUÇÃO}

Neste relatório iremos falar sobre a realização da 2ª fase do trabalho de laboratórios de informática III.
É semelhante ao trabalho realizado na primeira fase, à exceção que este é feito em \textit{Java}.
Abordaremos a estrutura utilizada, a maneira como chegamos à solução e os diferentes caminhos que percorremos para tentar minimizar o tempo de execução, bem como as dificuldades que foram surgindo.

\section{Parser}

Como o parse usado na primeira fase foi desenvolvido para \textit{Java},escolheu-se mais uma vez utilizar o parser do tipo \textit{SAX}. Este parser é orientado a eventos e recomendado pela documentação do \textit{libxml} para dar parse a grandes ficheiros.

\section{Modularidade}

\section{Classes e Estrutura}

\subsection{MainStruct}
Classe que contém a estrutura geral do projeto. Como concluímos na primeira fase que as tabelas de Hash era uma boa forma de organizar os dados,optamos novamente por essa estrutura.
\newline
Temos assim as seguintes variáveis de instância:
\begin{lstlisting}
 private HashMap<Long,Profile> profiles;
 private HashMap<Long,Post> posts;
 private HashMap<String,Long> tags;
 \end{lstlisting}

 \subsection{Profile}
 Classe que organiza a informação de um Perfil de um utilizador.
 \newline
 É composta pelas seguintes variáveis:

 \begin{lstlisting}
    private TreeSet<Post> posts;
    private String about_me;
    private String name;
    private long n_posts;
    private long id;
    private int reputation;
 \end{lstlisting}

 Onde \textit{posts} àrvore de Posts, \textit{aboutme} informação relativa ao perfil do Utilizador,\textit{name} Nome do User, \textit{nposts} Número de Posts,\textit{id} Id do User,\textit{reputation} Reputação do User.

 \subsection{Post}
 Tem apenas uma variável de instância que remete para o tipo de um post em que 1 é Question e 2 é Answer.

 \begin{lstlisting}
  private long type;
 \end{lstlisting}

 \newline

  \subsection{Answer}

  Classe que especifica a informação de um post em Respostas.
  \newline
  É composta pelas seguintes variáveis de instância:

 \begin{lstlisting}
   private LocalDate creation_date;
   private long id;
   private long owner_id;
   private long parent_id;
   private long comments;
   private long score;
 \end{lstlisting}

 Onde \textit{creationData} Data de criação de uma resposta, \textit{id} Id da resposta, \textit{ownerId} Id da pessoa que faz a resposta,\textit{parentId} Id da questão que está a ser respondida, \textit{comments} Número e comentários e \textit{score} Score de uma resposta.

 \subsection{Question}

 Classe que especifica a informação de um post em Questões.
 \newline
 É composta pelas variáveis de instância;

  \begin{lstlisting}
    private LocalDate creation_date;
    private ArrayList<Long> id_answers;
    private String title;
    private String tags;
    private long id;
    private long owner_id;
    private long n_answers;
    private long comments;
    private long score;
  \end{lstlisting}

  Onde \textit{creationDate} Data de criação de uma Questão,\textit{idAnswers} ArrayList Id das respostas de uma dada Questão,\textit{title} Título de uma questão,\textit{tags} Tag associada a uma Questão,\textit{id} Id da Questão, \textit{ownerId} Id da pessoa que faz a Questão, \textit{nAnswers} Número de respostas associadas a uma Questão, \textit{comments} Número de comentários, \textit{score} Score de uma Questão.

\section{Conclusão}

Esta segunda fase do projeto deu para aprender como deve ser feito um trabalho em Java onde devemos conseguir o menor tempo de execução.Tal como em C, devem ser implementadas estruturas eficientes,guardar apenas o necessário e não fazer cópias desnecessárias.
\newline
Em comparação à primeira fase, esta foi mais fácil.Não tivemos que criar tabelas de Hash ou listas ligadas,visto que o Java contém muitas estruturas pré definidas eficientes e relativamente mais fáceis de usar.Também não houve necessidade de libertar memória alocada, visto que o garbage collector faz isso por nós.
\newline
Também podemos aprender a usar streams do Java 8, apesar de serem ligeiramente mais lentas que os iteradores externos,tornam o código mais compacto,facilitando a leitura e o debug.
\newline
Tal como na primeira fase,consideramos que cumprimos os objetivos pretendidos.



\addtolength{\textheight}{-12cm}   % This command serves to balance the column lengths
                                  % on the last page of the document manually. It shortens
                                  % the textheight of the last page by a suitable amount.
                                  % This command does not take effect until the next page
                                  % so it should come on the page before the last. Make
                                  % sure that you do not shorten the textheight too much.

%%%%%%%%%%%%%%%%%%%%%%%%%%%%%%%%%%%%%%%%%%%%%%%%%%%%%%%%%%%%%%%%%%%%%%%%%%%%%%%%



%%%%%%%%%%%%%%%%%%%%%%%%%%%%%%%%%%%%%%%%%%%%%%%%%%%%%%%%%%%%%%%%%%%%%%%%%%%%%%%%



%%%%%%%%%%%%%%%%%%%%%%%%%%%%%%%%%%%%%%%%%%%%%%%%%%%%%%%%%%%%%%%%%%%%%%%%%%%%%%%%
%\section*{APPENDIX}

%Appendixes should appear before the acknowledgment.

%\section*{ACKNOWLEDGMENT}

%The preferred spelling of the word ÒacknowledgmentÓ in America is without an ÒeÓ after the ÒgÓ. Avoid the stilted expression, ÒOne of us (R. B. G.) thanks . . .Ó  Instead, try ÒR. B. G. thanksÓ. Put sponsor acknowledgments in the unnumbered footnote on the first page.



%%%%%%%%%%%%%%%%%%%%%%%%%%%%%%%%%%%%%%%%%%%%%%%%%%%%%%%%%%%%%%%%%%%%%%%%%%%%%%%%

%References are important to the reader; therefore, each citation must be complete and correct. If at all possible, references should be commonly available publications.



%\begin{thebibliography}{99}
%\bibitem{c1} G. O. Young, ÒSynthetic structure of industrial plastics (Book style with paper title and editor),Ó    in Plastics, 2nd ed. vol. 3, J. Peters, Ed.  New York: McGraw-Hill, 1964, pp. 15Ð64.
%\bibitem{c2} W.-K. Chen, Linear Networks and Systems (Book style).  Belmont, CA: Wadsworth, 1993, pp. 123Ð135.

%\end{thebibliography}




\end{document}
ss[letterpaper, 10 pt, conference]{IEEEtran}  % Comment this line out
\usepackage[utf8]{inputenc}
\usepackage[T1]{fontenc}                                                         % if you need a4paper
\usepackage[portuges]{babel}
%\documentclass[a4paper, 10pt, conference]{ieeeconf}      % Use this line for a4
 \usepackage{listings}
 \lstset { %
    language=Java,
}                                                         % paper
\usepackage{tikz}
\usepackage{xstring}
\usetikzlibrary{calc, arrows}
\usetikzlibrary{matrix,positioning,arrows.meta,arrows}

\tikzset{
mymat/.style={
  matrix of math nodes,
  text height=2.5ex,
  text depth=0.75ex,
  text width=3.25ex,
  align=center,
  column sep=-\pgflinewidth
  },
mymats/.style={
  mymat,
  nodes={draw,fill=#1}
  }
}
\IEEEoverridecommandlockouts                              % This command is only
                                                          % needed if you want to
                                                          % use the \thanks command
\overrideIEEEmargins
% See the \addtolength command later in the file to balance the column lengths
% on the last page of the document



% The following packages can be found on http:\\www.ctan.org
%\usepackage{graphics} % for pdf, bitmapped graphics files
%\usepackage{epsfig} % for postscript graphics files
%\usepackage{mathptmx} % assumes new font selection scheme installed
%\usepackage{times} % assumes new font selection scheme installed
%\usepackage{amsmath} % assumes amsmath package installed
%\usepackage{amssymb}  % assumes amsmath package installed

\title{\LARGE \bf
Laboratórios de Informática III - Projecto em Java
}

%\author{ \parbox{3 in}{\centering Huibert Kwakernaak*
%         \thanks{*Use the $\backslash$thanks command to put information here}\\
%         Faculty of Electrical Engineering, Mathematics and Computer Science\\
%         University of Twente\\
%         7500 AE Enschede, The Netherlands\\
%         {\tt\small h.kwakernaak@autsubmit.com}}
%         \hspace*{ 0.5 in}
%         \parbox{3 in}{ \centering Pradeep Misra**
%         \thanks{**The footnote marks may be inserted manually}\\
%        Department of Electrical Engineering \\
%         Wright State University\\
%         Dayton, OH 45435, USA\\
%         {\tt\small pmisra@cs.wright.edu}}
%}

\author{Adriana Meireles (a82582), Eduardo Jorge Barbosa (a83344), Filipe Monteiro (a80229)% <-this % stops a space
}


\begin{document}



\maketitle
\thispagestyle{empty}
\pagestyle{empty}


%%%%%%%%%%%%%%%%%%%%%%%%%%%%%%%%%%%%%%%%%%%%%%%%%%%%%%%%%%%%%%%%%%%%%%%%%%%%%%%%
\begin{abstract}

Este trabalho prático teve como objectivo replicar o anterior projeto (aplicado em C) mas agora em Java - a organização de grandes volumes de dados de forma a responder a questões, pré definidas, em tempo útil. Os objectos de estudo foram vários dumps do \textit{Stack Exchange}.

\end{abstract}


%%%%%%%%%%%%%%%%%%%%%%%%%%%%%%%%%%%%%%%%%%%%%%%%%%%%%%%%%%%%%%%%%%%%%%%%%%%%%%%%
\section{INTRODUÇÃO}

Neste relatório iremos falar sobre a realização da 2ª fase do trabalho de laboratórios de informática III.
É semelhante ao trabalho realizado na primeira fase, à exceção que este é feito em \textit{Java}.
\newline
O fundamental deste trabalho prático consistiu em converter as estruturas e querys existentes ,construídas no projeto anterior ,em estruturas mais dinâmicas, de mais flexibilidade e portabilidade, assim como melhoras de desempenho por parte de \textit{features} existentes na linguagem de programação Java.
\newline
Abordaremos a estrutura utilizada, a maneira como chegamos à solução e os diferentes caminhos que percorremos para tentar minimizar o tempo de execução, bem como as dificuldades que foram surgindo.

\section{Parser}

Como o parse usado na primeira fase foi desenvolvido para \textit{Java},escolheu-se mais uma vez utilizar o parser do tipo \textit{SAX}. Este parser é orientado a eventos e recomendado pela documentação do \textit{libxml} para dar parse a grandes ficheiros.

\section{Modularidade}
A estrutura do \textit{package} engine tem o seguinte diagrama:

\section{Classes e Estrutura}

\subsection{Main Struct}
Tal como no projeto em C, a estrutura principal é constituida por 3 HashMaps, uma para os perfis, outra para posts e outra para tags, e ainda a reformulado Tardis que contem os posts todos organizados pelo critério \textit{x}.
\newline
Temos assim as seguintes variáveis de instância:
\begin{lstlisting}
 private HashMap<Long,Profile> profiles;
 private HashMap<Long,Post> posts;
 private HashMap<String,Long> tags;
 private Tardis tardis64;
 \end{lstlisting}

\subsection{USER}

 Na Class de um Utilizador existem, entre os dados pessoais (nome, bio, ID,reputação) uma TreeSet de Posts
 pertencentes ao utilizador (ordenados naturalmente do mais recente - mais antigo).
 É composta pelas seguintes variáveis:

 \begin{lstlisting}
    private TreeSet<Post> posts;
    private String about_me;
    private String name;
    private long n_posts;
    private long id;
    private int reputation;
 \end{lstlisting}

 Nesta foram implementados apenas os métodos básicos (construtores, gets, sets, clone, toString).

\subsection{POST, QUESTION, ANSWER}

 A Class Post é uma Class Abstracta de onde duas classes extendem: Question e Answer.
 Foi utilizada uma class Abstrata para podermos tratar destas duas subclasses como se fossem
 "do mesmo tipo", facilitando assim o uso destas.

 Class Post:
 \begin{lstlisting}
 public abstract class Post {

    private long type;
 }

 public class Question extends Post {

     private LocalDateTime creation_date;
     private HashMap<Long,Answer> answers;
     private String title;
     private List<String> tags; //TODO Maybe we can change with java
     private long id;
     private long owner_id;
     private long n_answers;
     private long comments;
     private long score;
 }

 public class Answer extends Post {

     private LocalDateTime creation_date;
     private long id;
     private long owner_id;
     private long parent_id;
     private long comments;
     private long score;

 }
 \end{lstlisting}

 \subsection{TARDIS}

 \subsection{ESTRUTURAS AUXILIARES && EXCEPÇÕES}
 Para facilitar o manuseamento de algumas querys, decidiu-se criar e/ou recriar algumas classes já implementadas
 pelo Java, mas com pequenos acréscimos. Nomeadamente:

 \subsubsection{BoundedTreeSet}
 \begin{lstlisting}
 public class BoundedTreeSet<E> extends TreeSet<E> {
     private int limit;

     @Override
     public boolean add(E e){
         super.add(e);
         if(this.size() > this.limit){
         this.remove(this.last());
         }

         return true;
     }
 }
 \end{lstlisting}
 Esta TreeSet, funciona exatamente como uma TreeSet habitual, perfeita para inserções ordenadas, mas tem
 uma informação e cuidado extra: um limite de tamanho. Sempre que estiver no tamanho máximo insere ordenadamente
 e retira o último, mantendo sempre o seu tamanho controlado. Isto foi nos útil particularmente na \underline{Query 5}.
 \newline

 Em relação a comparadores, temos as seguintes:
 \begin{itemize}
     \item PostCreationDateComparator --> Compara dois Posts em relação á sua data de criação, de forma \textbf{Descendente}.
     \item ProfileNPostsComparator --> Compara dois Utilizadores em relacão ao número de posts, de forma \textbf{Descendente}.

    \item QuestionCreationDateComparator --> Compara duas Question em relação á sua data de criação, de forma \textbf{Descendente}.
 \end{itemize}


 Quanto a excepções, criamos as seguintes:
 \begin{itemize}
     \item NoPostFoundException --> Caso um ID nao exista entre os Posts.
     \item NoProfileFoundException --> Caso um ID nao exista entre os Perfis.
     \item NoTagFoundException --> Caso um ID nao exista entre as Tags.
 \end{itemize}

\section{Conclusão}

Esta segunda fase do projeto deu para aprender como deve ser feito um trabalho em Java onde devemos conseguir o menor tempo de execução.Tal como em C, devem ser implementadas estruturas eficientes,guardar apenas o necessário e não fazer cópias desnecessárias.
\newline
Em comparação à primeira fase, esta foi mais fácil.Não tivemos que criar tabelas de Hash ou listas ligadas,visto que o Java contém muitas estruturas pré definidas eficientes e relativamente mais fáceis de usar.Também não houve necessidade de libertar memória alocada, visto que o garbage collector faz isso por nós.
\newline
Também podemos aprender a usar streams do Java 8, apesar de serem ligeiramente mais lentas que os iteradores externos,tornam o código mais compacto,facilitando a leitura e o debug.
\newline
Tal como na primeira fase,consideramos que cumprimos os objetivos pretendidos.



\addtolength{\textheight}{-12cm}   % This command serves to balance the column lengths
                                  % on the last page of the document manually. It shortens
                                  % the textheight of the last page by a suitable amount.
                                  % This command does not take effect until the next page
                                  % so it should come on the page before the last. Make
                                  % sure that you do not shorten the textheight too much.

%%%%%%%%%%%%%%%%%%%%%%%%%%%%%%%%%%%%%%%%%%%%%%%%%%%%%%%%%%%%%%%%%%%%%%%%%%%%%%%%



%%%%%%%%%%%%%%%%%%%%%%%%%%%%%%%%%%%%%%%%%%%%%%%%%%%%%%%%%%%%%%%%%%%%%%%%%%%%%%%%



%%%%%%%%%%%%%%%%%%%%%%%%%%%%%%%%%%%%%%%%%%%%%%%%%%%%%%%%%%%%%%%%%%%%%%%%%%%%%%%%
%\section*{APPENDIX}

%Appendixes should appear before the acknowledgment.

%\section*{ACKNOWLEDGMENT}

%The preferred spelling of the word ÒacknowledgmentÓ in America is without an ÒeÓ after the ÒgÓ. Avoid the stilted expression, ÒOne of us (R. B. G.) thanks . . .Ó  Instead, try ÒR. B. G. thanksÓ. Put sponsor acknowledgments in the unnumbered footnote on the first page.



%%%%%%%%%%%%%%%%%%%%%%%%%%%%%%%%%%%%%%%%%%%%%%%%%%%%%%%%%%%%%%%%%%%%%%%%%%%%%%%%

%References are important to the reader; therefore, each citation must be complete and correct. If at all possible, references should be commonly available publications.



%\begin{thebibliography}{99}
%\bibitem{c1} G. O. Young, ÒSynthetic structure of industrial plastics (Book style with paper title and editor),Ó    in Plastics, 2nd ed. vol. 3, J. Peters, Ed.  New York: McGraw-Hill, 1964, pp. 15Ð64.
%\bibitem{c2} W.-K. Chen, Linear Networks and Systems (Book style).  Belmont, CA: Wadsworth, 1993, pp. 123Ð135.

%\end{thebibliography}




\end{document}
